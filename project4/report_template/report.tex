\documentclass[a4paper, 11pt]{article}
\usepackage{graphicx}
\usepackage{amsmath}
\usepackage[pdftex]{hyperref}

% Lengths and indenting
\setlength{\textwidth}{16.5cm}
\setlength{\marginparwidth}{1.5cm}
\setlength{\parindent}{0cm}
\setlength{\parskip}{0.15cm}
\setlength{\textheight}{22cm}
\setlength{\oddsidemargin}{0cm}
\setlength{\evensidemargin}{\oddsidemargin}
\setlength{\topmargin}{0cm}
\setlength{\headheight}{0cm}
\setlength{\headsep}{0cm}

\renewcommand{\familydefault}{\sfdefault}

\title{Introduction to Learning and Intelligent Systems - Spring 2015}
\author{jmohan@student.ethz.ch\\ nleow@student.ethz.ch\\ wongs@student.ethz.ch\\}
\date{\today}

\begin{document}
\maketitle

\section*{Project 4 : Classification with Missing Labels}

We used 75\% of the labelled data for training, leaving behind 25\% of the data for cross validation. To fill in the missing labels for the training data and develop a model for predicting the labels of new data, we used the \textit{LabelPropagation} library from \textit{sklearn}.

In order to improve the accuracy of our predictions, we trained 5 predictors based on 5 different random splits of our test data and took the average of the predicted probabilites by the predictors which scored less than 0.25 on our cross validation set.

\end{document}
